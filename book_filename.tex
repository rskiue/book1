% Options for packages loaded elsewhere
\PassOptionsToPackage{unicode}{hyperref}
\PassOptionsToPackage{hyphens}{url}
%
\documentclass[
]{book}
\title{book\_title}
\author{author\_name}
\date{2022-02-12}

\usepackage{amsmath,amssymb}
\usepackage{lmodern}
\usepackage{iftex}
\ifPDFTeX
  \usepackage[T1]{fontenc}
  \usepackage[utf8]{inputenc}
  \usepackage{textcomp} % provide euro and other symbols
\else % if luatex or xetex
  \usepackage{unicode-math}
  \defaultfontfeatures{Scale=MatchLowercase}
  \defaultfontfeatures[\rmfamily]{Ligatures=TeX,Scale=1}
\fi
% Use upquote if available, for straight quotes in verbatim environments
\IfFileExists{upquote.sty}{\usepackage{upquote}}{}
\IfFileExists{microtype.sty}{% use microtype if available
  \usepackage[]{microtype}
  \UseMicrotypeSet[protrusion]{basicmath} % disable protrusion for tt fonts
}{}
\makeatletter
\@ifundefined{KOMAClassName}{% if non-KOMA class
  \IfFileExists{parskip.sty}{%
    \usepackage{parskip}
  }{% else
    \setlength{\parindent}{0pt}
    \setlength{\parskip}{6pt plus 2pt minus 1pt}}
}{% if KOMA class
  \KOMAoptions{parskip=half}}
\makeatother
\usepackage{xcolor}
\IfFileExists{xurl.sty}{\usepackage{xurl}}{} % add URL line breaks if available
\IfFileExists{bookmark.sty}{\usepackage{bookmark}}{\usepackage{hyperref}}
\hypersetup{
  pdftitle={book\_title},
  pdfauthor={author\_name},
  hidelinks,
  pdfcreator={LaTeX via pandoc}}
\urlstyle{same} % disable monospaced font for URLs
\usepackage{longtable,booktabs,array}
\usepackage{calc} % for calculating minipage widths
% Correct order of tables after \paragraph or \subparagraph
\usepackage{etoolbox}
\makeatletter
\patchcmd\longtable{\par}{\if@noskipsec\mbox{}\fi\par}{}{}
\makeatother
% Allow footnotes in longtable head/foot
\IfFileExists{footnotehyper.sty}{\usepackage{footnotehyper}}{\usepackage{footnote}}
\makesavenoteenv{longtable}
\usepackage{graphicx}
\makeatletter
\def\maxwidth{\ifdim\Gin@nat@width>\linewidth\linewidth\else\Gin@nat@width\fi}
\def\maxheight{\ifdim\Gin@nat@height>\textheight\textheight\else\Gin@nat@height\fi}
\makeatother
% Scale images if necessary, so that they will not overflow the page
% margins by default, and it is still possible to overwrite the defaults
% using explicit options in \includegraphics[width, height, ...]{}
\setkeys{Gin}{width=\maxwidth,height=\maxheight,keepaspectratio}
% Set default figure placement to htbp
\makeatletter
\def\fps@figure{htbp}
\makeatother
\setlength{\emergencystretch}{3em} % prevent overfull lines
\providecommand{\tightlist}{%
  \setlength{\itemsep}{0pt}\setlength{\parskip}{0pt}}
\setcounter{secnumdepth}{5}
%lualatexjaを使うために,ヘッダーとして読み込ませるtexファイル
%必要なヘッダーはここに記述しておくと楽です
%個人的な趣味でIPAフォントを利用しています
\usepackage{luatexja}
\usepackage{luatexja-fontspec}
\setmainjfont{IPAGothic}
\setsansjfont{IPAGothic}
\ifLuaTeX
  \usepackage{selnolig}  % disable illegal ligatures
\fi

\begin{document}
\maketitle

{
\setcounter{tocdepth}{1}
\tableofcontents
}
\hypertarget{book_title}{%
\chapter{book\_title}\label{book_title}}

このRmdファイルを \texttt{bookdown::render\_book("index.Rmd")}すると,自動的に製本(?)します。

なお(私の考えうる限りで)最小構成で作ってます。実際に作ろうと思うなら,\href{https://bookdown.org/yihui/bookdown/}{本家ドキュメント}を参照してください。

以下は説明用の文章を貼り付けてます。\textbf{実際には削除してください}。

不明な点があれば,Twitterの{[}@kazutan{]}(\url{https://twitter.com/kazutan}) もしくはこのリポジトリのissue,あるいはr-wakalangのrmarkdownのチャンネルまでおねがいします。

\hypertarget{ux66f8ux7c4dux30d5ux30a1ux30a4ux30ebux306eux4f5cux6210ux65b9ux6cd5}{%
\section{書籍ファイルの作成方法}\label{ux66f8ux7c4dux30d5ux30a1ux30a4ux30ebux306eux4f5cux6210ux65b9ux6cd5}}

\hypertarget{ux5fc5ux8981ux306aux30d1ux30c3ux30b1ux30fcux30b8ux74b0ux5883ux306aux3069}{%
\subsection{必要なパッケージ,環境など}\label{ux5fc5ux8981ux306aux30d1ux30c3ux30b1ux30fcux30b8ux74b0ux5883ux306aux3069}}

Knitr, rmarkdown, bookdownのパッケージがデータのレンダリングに必要です。またpandocの新しいのが必要で,面倒でしたらRStudioの最新版をインストールしてください(内包してます)。
ggplot2逆引き記事内にて使用するパッケージも必要となります。おそらくggplot2パッケージぐらいで大丈夫だと思いますが,面倒でしたらtidyverseパッケージを導入してください。これをインストールするとHadleyverseなパッケージ群が自動的にインストールされます。
もしpdf bookを作りたいのであれば,マシンにtex環境が必要です。日本語のフォントにIPAフォントを指定していますので,以下からダウンロードしてください。

\url{http://ipafont.ipa.go.jp/}

また,bookdownはutf-8しか受け付けません。そのためwindowsではうまく動かないかもしれません(未検証)。もし何かありましたらissueなりkazutan までご連絡ください。

私の作業環境(動作確認環境)は,最後にまとめて表示しています。

\hypertarget{download}{%
\subsection{Download}\label{download}}

git cloneして持ってくるか,右側のDownload Zipで持ってきてください:

\begin{verbatim}
$ git clone git@github.com:kazutan/bookdown_ja-template.git
\end{verbatim}

\hypertarget{ux30ecux30f3ux30c0ux30eaux30f3ux30b0ux672cux306eux30d5ux30a1ux30a4ux30ebux4f5cux6210}{%
\subsection{レンダリング(本のファイル作成)}\label{ux30ecux30f3ux30c0ux30eaux30f3ux30b0ux672cux306eux30d5ux30a1ux30a4ux30ebux4f5cux6210}}

\hypertarget{ux7a2eux985e}{%
\subsubsection{種類}\label{ux7a2eux985e}}

\begin{itemize}
\tightlist
\item
  gitbook形式: 以下のコードを実行
\end{itemize}

\begin{verbatim}
bookdown::render_book("index.Rmd", output_format = "bookdown::gitbook")
\end{verbatim}

\begin{itemize}
\tightlist
\item
  epub形式: 以下のコードを実行
\end{itemize}

\begin{verbatim}
bookdown::render_book("index.Rmd", output_format = "bookdown::epub_book")
\end{verbatim}

\begin{itemize}
\tightlist
\item
  pdf形式: 以下のコードを実行
\end{itemize}

\begin{verbatim}
bookdown::render_book("index.Rmd", output_format = "bookdown::pdf_book")
\end{verbatim}

RStudioを利用しているなら,BuildパネルでBuild Bookから選択してください。もしBuildタブがRStudioで表示されない場合,一度RStudioを終了させてもう一度開いてください。

\hypertarget{ux751fux6210ux7269ux306eux5834ux6240}{%
\subsection{生成物の場所}\label{ux751fux6210ux7269ux306eux5834ux6240}}

生成物は,\texttt{\_book}ディレクトリに置かれるように設定してます。\texttt{.epub}と\texttt{.pdf}は単独ファイルで,それ以外はgitbook形式のファイルとなります。

\hypertarget{session-info}{%
\section{session info}\label{session-info}}

\begin{verbatim}
Session info -----------------------------------------------------------------------------------
 setting  value                       
 version  R version 3.3.2 (2016-10-31)
 system   x86_64, linux-gnu           
 ui       RStudio (1.0.44)            
 language (EN)                        
 collate  en_US.UTF-8                 
 tz       <NA>                        
 date     2016-11-12                  

Packages ---------------------------------------------------------------------------------------
 package    * version  date       source                            
 backports    1.0.4    2016-10-24 cran (@1.0.4)                     
 bookdown     0.1.18   2016-11-08 Github (rstudio/bookdown@601437d) 
 devtools     1.12.0   2016-06-24 CRAN (R 3.3.1)                    
 digest       0.6.10   2016-08-02 cran (@0.6.10)                    
 evaluate     0.10     2016-10-11 CRAN (R 3.3.1)                    
 htmltools    0.3.5    2016-03-21 CRAN (R 3.3.1)                    
 httpuv       1.3.3    2015-08-04 CRAN (R 3.2.3)                    
 knitr        1.15     2016-11-09 CRAN (R 3.3.2)                    
 magrittr     1.5      2014-11-22 CRAN (R 3.2.3)                    
 memoise      1.0.0    2016-01-29 CRAN (R 3.2.3)                    
 mime         0.5      2016-07-07 CRAN (R 3.3.1)                    
 miniUI       0.1.1    2016-01-15 cran (@0.1.1)                     
 R6           2.2.0    2016-10-05 CRAN (R 3.3.1)                    
 Rcpp         0.12.7   2016-09-05 CRAN (R 3.3.1)                    
 rmarkdown    1.1.9014 2016-11-08 Github (rstudio/rmarkdown@91c7de2)
 rprojroot    1.1      2016-10-29 cran (@1.1)                       
 rstudioapi   0.6      2016-06-27 CRAN (R 3.3.1)                    
 shiny        0.14.2   2016-11-01 cran (@0.14.2)                    
 stringi      1.1.2    2016-10-01 CRAN (R 3.3.1)                    
 stringr      1.1.0    2016-08-19 CRAN (R 3.3.2)                    
 withr        1.0.2    2016-06-20 CRAN (R 3.3.1)                    
 xtable       1.8-2    2016-02-05 CRAN (R 3.2.3)                    
 yaml         2.1.13   2014-06-12 CRAN (R 3.2.3)     
\end{verbatim}

\hypertarget{ux7ae0ux306eux30bfux30a4ux30c8ux30ebux3092ux3053ux3053ux306bux5165ux529b}{%
\chapter{章のタイトルをここに入力}\label{ux7ae0ux306eux30bfux30a4ux30c8ux30ebux3092ux3053ux3053ux306bux5165ux529b}}

進捗どうですか?aa

適当に編集してください。

\hypertarget{ux7ae0ux306eux30bfux30a4ux30c8ux30eb2}{%
\chapter{章のタイトル2}\label{ux7ae0ux306eux30bfux30a4ux30c8ux30eb2}}

進捗どうですか?

\hypertarget{ux7bc0ux898bux51faux30571}{%
\section{節見出し1}\label{ux7bc0ux898bux51faux30571}}

ほげほげ

\hypertarget{ux7bc0ux898bux51faux30572}{%
\section{節見出し2}\label{ux7bc0ux898bux51faux30572}}

ふがふが

\end{document}
